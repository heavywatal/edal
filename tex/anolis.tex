\documentclass{article}
\usepackage{graphicx,verbatim,amsmath,epsfig,fancyhdr}
\usepackage[comma,sort]{natbib}
%\usepackage{indxcite}
%\usepackage[notcite]{showkeys}  % to have keys in margins
%\usepackage{showkeys}  % to have keys in margins

%\usepackage{cite}


\topmargin -.25in
\setlength{\textheight}{8.25in}
\setlength{\oddsidemargin}{0.1in}
\setlength{\voffset}{0in}
\setlength{\textwidth}{6.3in}

\setcounter{secnumdepth}{0}
\pagestyle{fancy}

\lhead[\today]{\today}
\rhead[\thepage]{\thepage}
\cfoot{}

%\citestyle{nature}    % removes parentheses everywhere
%\citestyle{bio}     % see changes in /usr/share/texmf/tex/latex/natbib/natbib.sty

%\citeindextrue
%\makeindx

%\bibliographystyle{siam}
%\bibliographystyle{alpha}
%\bibliographystyle{acm}
%\bibliographystyle{nature}
%\bibliographystyle{unsrt}	% unsorted; in the order of appearance; numerical labels
%\bibliographystyle{bioessays}
\bibliographystyle{apalike}
%\bibliographystyle{cbe}
%\nocite{*}



\begin{document}           % End of preamble and beginning of text.

%\renewcommand{\@cite}[2]{ {#1\if@tempswa , #2\fi}}
%\renewcommand{\@cite}[2]{[{#1\ifthenelse{\boolean{@tempswa}}{,#2}{}]}
%\renewcommand{\@cite}[2]{*{#1\if@tempswa , #2\fi}*}  % for text-labels
%\renewcommand{\@biblabel}[1]{}
%\renewcommand{\@biblabel}[1]{*#1*}    %
%\citeleft{*}
%\citeright{*}		% to change parenthesis; didn't work



\renewcommand{\baselinestretch}{1}
\normalsize


%LRH: SERGEY GAVRILETS\\

%RRH: MODELS OF SPECIATION\\

\title{Speciation in {\em Anolis}: model description}
% \author{Sergey Gavrilets and Aaron Vose\\
% Department of Ecology and Evolutionary Biology,\\
% Department of Mathematics, University of Tennessee,\
% Knoxville, TN 37996, USA.\\
% phone: 865-974-8136,\
% fax: 865-974-3067,\
% email: sergey@tiem.utk.edu}

\maketitle


\section{Space}

Space is subdivided into a rectangular array of ``patches''. There are
$2$ environmental characteristics corresponding to the height ($u$) at
which a lizard occurs and the diameter ($v$) of the object upon which
they are sitting. Each patch is characterized by a frequency
distribution $F(u,v)$. We can consider 2 possibilities.

{\em Old version}.\quad
The distribution of $u$ is normal with mean $\mu$ and variance $\sigma^2$, 
and $v$ has an exponential distribution with mean
$\theta^{-1} = c_0 - c_1 u$, where $c_0$ and $c_1$ are positive
constants.  In particular, the natural constraint $\theta^{-1} > 0$
implies $1 \le c_0/c_1$.  The joint density $F$ is
\begin{equation} \label{gaus}
F(u,v) \; = \;
k \,\exp (- \sigma (u - \mu)^2 )\exp\Big(\!- \theta  v \Big) \theta,
\end{equation}
where $k$ is normalizing constant.
We can truncate this distribution so it's defined for $0\leq u,v \leq 1$.
We can also set the mean of $u$ at $\mu=1/2$.


{\em New version}.\quad Height $u$ has a symmetric beta distribution mean 1/2
and with parameter
$\alpha>1$ and the diameter has a triangular distribution on the interval $[0,1-u]$
so that the joint density is 
\begin{equation} \label{beta}
F(u,v) \; = \;
\frac{\Gamma(2\alpha)}{\Gamma(\alpha)^2} (u(1-u))^{\alpha-1}\ \frac{2}{1-u}(1-\frac{v}{1-u}),
\end{equation}
where $\Gamma$ is gamma function. The mean of $u$ is 1/2 and the variance is $1/(4(1+2\alpha))$.
For any given $u$, the mean of $v$ is $(1-u)/3$.
This distribution is defined for $0 \leq u \leq 1$ and $0 \leq v \leq 1-u$.
The triangular distribution seems to be a good approximation of the exponential
distribution (good for our purposes).

\section{Individuals}

Individuals are sexual and diploid. Each individual has a number of
additive quantitative characters:
\begin{itemize}
\item 2 ``morphological'' characters $x_0$ (toepad size) and $x_1$
  (limb length) that control viability;
\item 2 ``habitat preference'' characters: the most preferred height
  $y_0$ and the most preferred diameter $y_1$
%   , tolerance for height $z_0$, and tolerance for diameter $z_1$; 
\item three ``mating compatibility'' characters $m, f$, and $c$.
\end{itemize}

The male display trait $m$ is expressed in males only, whereas female
mating preference $f$ and tolerance $c$ are expressed in females
only. Other traits are expressed in both sexes.  All traits are scaled
to be between 0 and 1 and are controlled by different unlinked
diallelic loci with equal effects. Mutations occur at equal rates
across all loci; the probabilities of forward and backward mutations
are equal.


\newpage
\section{Habitat preference and carrying capacity}

{\bf Habitat preference}.\quad
The relative preference of an individual with habitat preference
traits $y_0,y_1$ for environmental characteristic $(u,v)$ is 
	\begin{equation}
	\Xi(y_0,y_1\mid u,v) \sim \exp(-h_0(u-y_0)^2-h_1(v-y_1)^2)
	\end{equation}
We can also consider a quadratic approximation:
	\begin{equation} \label{quad}
	\Xi(y_0,y_1\mid u,v) \sim 1-h_0(u-y_0)^2-h_1(v-y_1)^2.
	\end{equation}

{\bf Viability}.\quad Let viability of genotype $(x_0,x_1)$ in niche $(u,v)$ be Gaussian,
\begin{equation}
		W(x_0,x_1 \mid u,v) = \exp(-s_0(u-x_0)^2-s_1(v-x_1)^2),
\end{equation}
where the $\sigma_{s_i}>0$ are parameters measuring the strength of
selection.  Note that the optimum values of traits $x_0$ and $x_1$
under environmental conditions $(u,v)$ are $x_0 = u$ and $x_1 = v$.\\

{\bf Carrying capacity}.\quad Consider a specific genotype that has an absolute
preference for environment $(u,v)$ (so that $\Xi$ is a $\delta$-function). Then a natural
measure of the carrying capacity of a population of such individuals is
\[
    K_0 F(u,v) W(x_0,x_1 \mid u,v),
\]
where $K_0$ is a scaling parameter, and $K_0 F(u,v)$ can be thought of as the maximum
number of individuals with an absolute preference for environment $(u,v)$ is they
also have perfect adaptation for this niche (i.e. if $W(x_0,x_1 \mid u,v)=1$).
Naturally, the more common environments can maintain higher densities of individuals
that prefer them. Now,
\begin{equation}
    K_e(I)= K_0 \int \int F(u,v) W(x_0,x_1 \mid u,v) \Xi(y_0,y_1\mid u,v) du dv
\end{equation}
is the average carrying capacity of genotype $(x_0,x_1, y_0, y_1)$.


\subsubsection{Effective number of competitors}

Following Roughgarden and others, we define the competition coefficient for individual $I$ and $J$
as
\begin{subequations}
\begin{equation}
c(I,J)=exp( -c_0(y_{0,I}-y_{0,J})^2-c_1(y_{1,I}-y_{1,J})^2).
\end{equation}
That is, competition strength is evaluated  on the basis of their habitat preferences.
For individuals with identical preferences, $c(I,J)=1$, that is competition is the strongest.

For individual $I$, the effective number of competitors is 
\begin{equation}
N_e(I)=\sum_J c(I,J).
\end{equation}
\end{subequations}

\subsubsection{Survival}

The probability that an individual survives to the age of reproduction is
	\begin{equation} \label{BH}
		w(I)=\frac{1}{1+(b-1)\frac{N_e(I)^\alpha}{K_e(I)}},
	\end{equation}
where the parameter $\alpha$ controls the strength of
crowding, and $b>0$ is a parameter (average number of offspring per female;
see below).  Equation~(\ref{BH}) is the Beverton-Holt model which
represents a discrete-time analog of the logistic model. If $N_e(I)\ll
K_e(I)$, then $w(I)=1$ (survival probability is one), and if
$N_e(I)=K_e(I)$, then $w(I)=1/b$.

\section{Mating}

\subsubsection{Mating preferences}

The relative probability of mating between a
female with traits $f$ and $c$ and a male with trait $m$ is
	\begin{equation}  \label{pref}
	\psi(f,c\mid m) =  \left\{ \begin{array}{ll}
                         \exp \left( -(2c-1)^2 \frac{(f-m)^2}{2\sigma_a^2}\right) & \mbox{if}\ c > 0.5, \\
			  1 & \mbox{if}\ c=0.5,\\
                         \exp \left( -(2c-1)^2 \frac{(f-(1-m))^2}{2\sigma_a^2}\right) & \mbox{if}\ c<0.5,
                      \end{array} \right.
	\end{equation}
where parameter $\sigma_a$ scales the strength of female mating preferences. Under this parameterization,
females with $c=0.5$ mate randomly, females with $c>0.5$ prefer males whose trait $m$ is close to 
the female's trait $f$ (positive assortative mating), and females with $c<0.5$ prefer males whose 
trait $m$ is close to $1-f$ (negative assortative mating). 


\subsubsection{Mating frequencies}

The pseudo probability that female
$I = \langle f,c,\ldots \rangle$ mates with male $I^{\prime}= \langle m,\ldots \rangle$ is
\begin{equation}
P(I,I^{\prime}) = \psi(I,I^{\prime})\ C(I,J) \\
\end{equation}
This is the preference of $I$ for $I^{\prime}$, multiplied by their
correlation (with respect to habitat preferences $C(I,J)$).  These ``pseudo
probabilities'' are scaled to sum to one (by summing over the available males $I^\prime$).


\subsubsection{Offspring production}

Each mating results in a number of offspring drawn from a Poisson distribution with parameter $b$.
We assume that all adult females mate. This assumption implies that any costs of mate choice,
which can easily prevent divergence and speciation, are absent.
This assumption also means that the effective population size is increased relative to the actual
number of adults. 
\section{Dispersal}


With probability $m>0$, each offspring becomes a ``migrant.''
Each migrant goes to one of the 8 neighboring patches.
For patches at the boundary, the probability $m$ is reduced according 
to the number of neighbors they have.


\section{Life cycle}

There are two options. (1) viability selection $\rightarrow$ dispersal
$\rightarrow$ mating and offspring production, and (2) dispersal
$\rightarrow$ viability selection $\rightarrow$ mating and offspring production.

{\footnotesize [Q: which one is more appropriate for lizards? Jonathan:
I think dispersal before selection is more appropriate, although we actually know little about dispersal; still, the best bet is that it occurs when they are young.]}

{\bf So}, we consider the second option only.


\section{Initial conditions}

$K_{init}$ offsprings populate each patch in a small square of patches in the upper left corner.
All individuals are identical homozygotes such that all traits are exactly in the
middle of their range (i.e., $x_i=y_i=0.5$ for all $i$).
These initial conditions correspond to the colonization of a system by a small number of
genetically similar, poorly adapted generalists.

\end{document}
